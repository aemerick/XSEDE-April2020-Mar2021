\documentclass[12pt]{article} % font size here: some require 11 or 12 point
\usepackage{setspace}
\usepackage{fancyvrb}
\usepackage{epsfig}
\usepackage{fullpage}
\usepackage[small,compact]{titlesec}
\usepackage{times} 
\usepackage{enumitem}
\usepackage{wrapfig}
\setlength{\topmargin}{+0.0in}   %%%%%%%%% this is hacked (from +.0.1in) so that it looks right when converted.
\setlength{\oddsidemargin}{-0.0in}
\setlength{\evensidemargin}{-0.0in}
\setlength{\textheight}{9.0in}
\setlength{\textwidth}{6.5in}
\usepackage[margin=1.0in,headheight=0pt,footskip=20pt,headsep=0.2in]{geometry}
\usepackage{natbib}
% ----------------------------------------------------

\usepackage{natbib}
\usepackage{color}
\bibliographystyle{apj}
\newcommand{\apj}{ApJ}
\newcommand{\apss}{Astrophysics and Space Science}
\newcommand{\aj}{AJ}
\newcommand{\apjl}{ApJL}
\newcommand{\mnras}{MNRAS}
\newcommand{\apjs}{ApJS}
\newcommand{\pasp}{PASP}
\newcommand{\araa}{ARA\&A}
\newcommand{\aap}{A\&A}
\newcommand{\aaps}{A\&AS}
\newcommand{\pasj}{PASJ}
\newcommand{\prd}{Phys. Rev. D}
\newcommand{\nat}{Nature}
\newcommand{\nar}{New Astronomy Review}
\newcommand{\ao}{Applied Optics}
\newcommand{\physrep}{Physics Reports}
\newcommand{\etal}{et al}
\def\subsun{\mbox{$_{\odot}$}}
\def\lesssim{\mathrel{\hbox{\rlap{\hbox{%
 \lower4pt\hbox{$\sim$}}}\hbox{$<$}}}}
\def\gtrsim{\mathrel{\hbox{\rlap{\hbox{%
 \lower4pt\hbox{$\sim$}}}\hbox{$>$}}}}
\RequirePackage{natbib}
%\usepackage{setspace}
\setlength{\bibsep}{0.0pt}
\newenvironment{itemize*}%
{\begin{itemize}%
  \setlength{\itemsep}{0pt}%
    \setlength{\parskip}{0pt}}%
{\end{itemize}}

\newenvironment{enumerate*}%
{\begin{enumerate}%
  \setlength{\itemsep}{0pt}%
    \setlength{\parskip}{0pt}}%
{\end{enumerate}}

\usepackage{fancyhdr}
\pagestyle{fancy}


\usepackage{fancyhdr}
\pagestyle{fancy}
\lhead{Andrew Emerick}
\rhead{IAU PhD Prize - Thesis Summary}

%\usepackage[pagestyles]{titlesec}
%\newpagestyle{mystyle}{\sethead{}{}{Name -- Number}\setfoot{}{\thepage}{}}
%\pagestyle{mystyle}

\title{\vspace{-5ex} \Large Stellar Feedback and Chemical Evolution in Dwarf Galaxies
       \vspace{-2ex}}
\author{Andrew Emerick}
\date{\vspace{-5ex}}
\begin{document} \thispagestyle{empty}

% Star-by-Star Stellar Feedback and Chemical Evolution in Dwarf Galaxies

\maketitle

{\noindent \large \textbf{Abstract}}
\\
\\
While models of stellar feedback have been used as essential components in galaxy-scale simulations to match general properties of observed galaxies, there remains enormous uncertainty in how stellar feedback couples to the interstellar medium (ISM) to regulate star formation and drive outflows. Galactic chemical evolution -- probed by multi-element gas-phase and stellar abundances -- is a rich new domain with which to test simulations of galaxy evolution, particularly with the growing number and quality of detailed stellar abundances in the Milky Way and Local Group galaxies. To better understand stellar feedback and galactic chemical evolution, I developed the first set of galaxy-scale hydrodynamics simulations that follow the formation of individual stars and their associated stellar feedback and multi-element metal enrichment. Using this star-by-star method I follow the feedback and enrichment from stellar winds, stellar radiation, and supernovae in simulations of isolated, low-mass dwarf galaxies comparable to low-mass, star forming dwarf galaxies in the Local Group. For the first time, I demonstrate that metal-mixing in the ISM of these galaxies is non-uniform and varies metal-by-metal with the source of enrichment. This has important implications for our ability to use semi-analytic models to extract information about galactic chemical evolution from observed stellar abundance patterns.


\end{document}
